\documentclass[a4paper, 12pt]{article}

\usepackage[utf8]{inputenc}

\usepackage{graphicx}
\usepackage{caption}
\usepackage{parskip}

\usepackage{physics}
\usepackage{amsmath}

\usepackage{tikz}
\usepackage{tkz-graph}
\usepackage{tikzscale}
\usetikzlibrary{arrows}

\captionsetup{width=0.8 \linewidth}

\begin{document}

\title{\vspace{-6em}\textbf{Homework 4}\\ \Large Information Theory for Complex Systems \vspace{-3.2em} }
\author{}
\date{Februari - 2023}

\maketitle

\pagenumbering{gobble}

There are 9 possible probabilities, which through symmetries can be reduced down 3.
\begin{center}
    \begin{tabular}{c|c|c|c}
         & A & B & C \\ \hline
        A & $P(AA) = p_1$ & $P(AB) = p_2$ & $P(AC) = p_2$ \\ \hline
        B & $P(BA) = p_2$ & $P(BB) = p_3$ & $P(BC) = p_3$ \\ \hline
        C & $P(CA) = p_2$ & $P(CB) = p_3$ & $P(CC) = p_3$ \\
    \end{tabular}
\end{center}
Since energy is only determined by interactions in pairs we can conclude that finding a maximum of $\Delta S_2$ should be enough to determine $p_1,p_2$ and $p_3$.

Attempt a Lagrange optimization
\begin{equation}
    \begin{split}
        L(p_1, p_2, p_3, \beta, \mu) = \Delta S_2 & + \beta ( u - J(4p_2 - p_1))  \\
        & + \mu ( 1 - p_1 - 4 p_2 - 4 p_3 )
    \end{split}
\end{equation}
where we know everything except for an expression for the block entropy difference. Now we can calculate the block entropy difference
\begin{equation}
    \begin{array}{ll}
        \Delta S_2 & = \sum_{z_{n-1}} p(z_{n-1}) \sum_{z_n} p(z_n|z_{n-1}) \log \frac{1}{p(z_n|z_{n-1})} = \\
        & = - ( a p_1 \log p_1 + [2a + P(B) + P(C)] p_2 \log p_2  \\
        & \hspace{1em} + [2P(B) + 2P(C)] p_3 \log p_3 )
    \end{array}
\end{equation}
In this expression the only probabilities missing are $P(B)$ and $P(C)$. Since $B$ and $C$ appear in equal amounts of pairs with the same probabilities, the amounts of $B$ and $C$ must also be the same in the limit towards infinity. Therefore they split the proportion remaining after the $A$'s,
\begin{equation}
    P(B) = P(C) = \frac{1-a}{2}.
\end{equation}
Now we can simplifiy the block entropy further.
\begin{equation}
    \Delta S_2 =  - \left( a p_1 \log p_1 + [a + 1] p_2 \log p_2 + 2 [ 1 - a] p_3 \log p_3 \right)
\end{equation}
To obtain the equilibrium distribution from here, complete the Lagrange optimization since all values except the parameters to be optimized for ($p_1, p_2, p_3$) are known.

In the limit of zero temperature, the energy of the system would be as low as possible. This entails pairing as many spins as possible to collect the maximum amount of $-J$ contributions. The chain would be organised in blocks, containing only one spin with some crystal imperfections where two blocks of differing spin meets. Described by the probabilities $p_2$ will tend towards 0 (but never quite reach it since there must be some imperfections unless $a=1$). $p_1$ would approach $a$ and in turn $p_3$ would approach $(1 - p_1)/2 = (1-a)/2$.

\end{document}
