\documentclass[a4paper, 12pt]{article}

\usepackage[utf8]{inputenc}

\usepackage{graphicx}
\usepackage{caption}
\usepackage{parskip}

\usepackage{physics}
\usepackage{amsmath}

\usepackage{tikz}
\usepackage{tkz-graph}
\usepackage{tikzscale}
\usetikzlibrary{arrows}

\captionsetup{width=0.8 \linewidth}

\begin{document}

\title{\vspace{-6em}\textbf{Homework 4}\\ \Large Information Theory for Complex Systems \vspace{-3.2em} }
\author{}
\date{Februari - 2023}

\maketitle

There are 9 possible probabilities, which through symmetries can be reduced down 4.
\begin{center}
    \begin{tabular}{c|c|c|c}
         & A & B & C \\ \hline
        A & $P(AA) = p_1$ & $P(AB) = p_2$ & $P(AC) = p_2$ \\ \hline
        B & $P(BA) = p_3$ & $P(BB) = p_4$ & $P(BC) = p_4$ \\ \hline
        C & $P(CA) = p_3$ & $P(CB) = p_4$ & $P(CC) = p_4$ \\
    \end{tabular}
\end{center}
Since energy is only determined by interactions in pairs we can conclude that finding a maximum of $\Delta S_2$ should be enough to determine $p_1,p_2,p_3$ and $p_4$.

Attempt a Lagrange optimization
\begin{equation}
    \begin{split}
        L(p_1, p_2, p_3, p_4, \beta, \mu) = \Delta S_2 & + \beta ( u - J(2p_2 + 2p_3 - p_1))  \\
        & + \mu ( 1 - p_1 - 2 p_2 -  2p_3 - 4 p_4 )
    \end{split}
\end{equation}

(is  $p_3 = p_2$ ?)


\noindent\rule{\textwidth}{0.4pt}

$U$ is the expectation value of the internal energy.
\begin{gather}
    S[P] = \sum_i p_i \ln \frac{1}{p_i} \\
    \sum_i p_i h(i) = U \\
    \sum_i p_i f_k(i) = n_k, (k = 1, \dots , M) \\
    \sum_i p_i = 1
\end{gather}

\end{document}
