\documentclass[a4paper, 12pt]{article}

\usepackage[utf8]{inputenc}

\usepackage{graphicx}
\usepackage{caption}
\usepackage{parskip}

\usepackage{physics}
\usepackage{amsmath}

\usepackage{tikz}
\usepackage{tkz-graph}
\usepackage{tikzscale}
\usetikzlibrary{arrows}

\captionsetup{width=0.8 \linewidth}

\begin{document}

\title{\vspace{-6em}\textbf{Homework 4}\\ \Large Information Theory for Complex Systems \vspace{-3.2em} }
\author{}
\date{Februari - 2023}

\maketitle

There are 9 possible probabilities, which through symmetries can be reduced down 3.
\begin{center}
    \begin{tabular}{c|c|c|c}
         & A & B & C \\ \hline
        A & $P(AA) = p_1$ & $P(AB) = p_2$ & $P(AC) = p_2$ \\ \hline
        B & $P(BA) = p_2$ & $P(BB) = p_3$ & $P(BC) = p_3$ \\ \hline
        C & $P(CA) = p_2$ & $P(CB) = p_3$ & $P(CC) = p_3$ \\
    \end{tabular}
\end{center}
Since energy is only determined by interactions in pairs we can conclude that finding a maximum of $\Delta S_2$ should be enough to determine $p_1,p_2$ and $p_3$.

Attempt a Lagrange optimization
\begin{equation}
    \begin{split}
        L(p_1, p_2, p_3, \beta, \mu) = \Delta S_2 & + \beta ( u - J(4p_2 - p_1))  \\
        & + \mu ( 1 - p_1 - 4 p_2 - 4 p_3 )
    \end{split}
\end{equation}
where we know everything except for an expression for the block entropy difference. To construct an expression for $\Delta S_2$ we need to know the probabilities of the spin states themselves
\begin{equation}
    P(A) = ?, P(B) = ?, P(C) = ?.
\end{equation}
Now we can calculate the block entropy difference
\begin{equation}
    \begin{array}{ll}
        \Delta S_2 & = \sum_{z_{n-1}} p(z_{n-1}) \sum_{z_n} p(z_n|z_{n-1}) \log \frac{1}{p(z_n|z_{n-1})} = \\
        & = P(A) \left(P(AA) \log \frac{1}{P(AA)} + P(AB) \log \frac{1}{P(AB)} +P(AC) \log \frac{1}{P(AC)} \right) + \\
        & + \hspace{.23em} P(B) \left(P(BA) \log \frac{1}{P(BA)} + P(BB) \log \frac{1}{P(BB)} +P(BC) \log \frac{1}{P(BC)} \right) +  \\
        & + \hspace{.23em} P(C) \left(P(CA) \log \frac{1}{P(CA)} + P(CB) \log \frac{1}{P(CB)} +P(CC) \log \frac{1}{P(CC)} \right) = \\
        & = - P(A) \left( p_1 \log p_1 + p_2 \log p_2 + p_2 \log p_2  \right) \\
        & \hspace{1em} - P(B)  \left( p_2 \log p_2 + p_3 \log p_3 + p_3 \log p_3  \right) \\
        & \hspace{1em} - P(C)  \left( p_2 \log p_2 + p_3 \log p_3 + p_3 \log p_3  \right) = \\
        & = - ( a p_1 \log p_1 + [2a + P(B) + P(C)] p_2 \log p_2  \\
        & \hspace{1em} + [2P(B) + 2P(C)] p_3 \log p_3 )
    \end{array}
\end{equation}


\noindent\rule{\textwidth}{0.4pt}

$U$ is the expectation value of the internal energy.
\begin{gather}
    S[P] = \sum_i p_i \log \frac{1}{p_i} \\
    \sum_i p_i h(i) = U \\
    \sum_i p_i f_k(i) = n_k, (k = 1, \dots , M) \\
    \sum_i p_i = 1
\end{gather}

\end{document}
